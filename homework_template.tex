% Copyright (C) 2024 Nicholas Fleischhauer
%
% This homework template is free software: you can redistribute it and/or modify
% it under the terms of the GNU General Public License as published by
% the Free Software Foundation, either version 3 of the License, or
% (at your option) any later version.
%
% This template is distributed in the hope that it will be useful,
% but WITHOUT ANY WARRANTY; without even the implied warranty of
% MERCHANTABILITY or FITNESS FOR A PARTICULAR PURPOSE. See the
% GNU General Public License for more details.
%
% You should have received a copy of the GNU General Public License
% along with this template. If not, see <https://www.gnu.org/licenses/>.

\documentclass[11pt, a4paper, oneside]{article}

\title{Homework 1}
\author{Nicholas Fleischhauer}
\date{\today}

% Encoding and language support
\usepackage[utf8]{inputenc}
\usepackage[english]{babel}

% Mathematical packages
\usepackage{amsmath, amssymb, amsthm}

% Graphics, tables, and layout
\usepackage{graphicx}
\usepackage{float}
\usepackage[inner=1.5cm, outer=1.5cm, top=2.5cm, bottom=2.5cm]{geometry}
\graphicspath{{assets/}}

% Algorithms
\usepackage{algorithm, algpseudocode}

% Colors and fancy text
\usepackage[dvipsnames, svgnames, table]{xcolor}
\usepackage{fancyhdr}

\setlength{\headheight}{13.6pt}

% Custom colors
\definecolor{darkblue}{rgb}{0,0,.6}
\definecolor{darkred}{rgb}{.7,0,0}
\definecolor{darkgreen}{rgb}{0,.6,0}
\definecolor{red}{rgb}{.98,0,0}
\definecolor{OliveGreen}{cmyk}{0.64,0,0.95,0.40}
\definecolor{CadetBlue}{cmyk}{0.62,0.57,0.23,0}
\definecolor{lightlightgray}{gray}{0.93}

% Page footer management
\usepackage{lastpage}

% For code blocks (minted - requires pygments)
\usepackage{minted}

% Quote handling (loaded after minted to avoid warnings)
\usepackage{csquotes}

% Alternative code listing package (listings)
\usepackage{listings}

% Bibliography management
\usepackage[backend=biber, style=authoryear, citestyle=authoryear]{biblatex}
\addbibresource{references.bib} % Replace with your actual .bib file name
\lstset{
  language=Python,
  basicstyle=\ttfamily\small,
  keywordstyle=\color{OliveGreen},
  commentstyle=\color{gray},
  numbers=left,
  numberstyle=\tiny,
  stepnumber=1,
  numbersep=5pt,
  backgroundcolor=\color{lightlightgray},
  frame=none,
  tabsize=2,
  captionpos=t,
  breaklines=true,
  showspaces=false,
  showtabs=false,
  columns=flexible,
  morecomment=[l][\color{magenta}]{\#},
  morekeywords={__global__, __device__}
}

% For paragraph formatting
\usepackage{parskip}
\setlength{\parskip}{10pt plus 1pt minus 1pt}

% Hyperlinks
\usepackage{hyperref}
\hypersetup{
    pdfauthor={Nicholas Fleischhauer},
    pdftitle={Homework Title},
    pdfsubject={Homework Assignment},
    pdfkeywords={Homework, LaTeX, Template},
    pdfproducer={GNU General Public License v3.0 or later, see https://www.gnu.org/licenses/gpl-3.0.en.html},
    colorlinks=true,
    linkcolor=darkred,
    filecolor=magenta,
    urlcolor=darkblue,
    citecolor=darkblue,
    bookmarksnumbered=true,
    plainpages=false
}

% Verbatim and special text environments
\usepackage{verbatim, fancyvrb}
\VerbatimFootnotes

% Page header and footer style
\pagestyle{fancy}
\fancyhf{}
\lhead{Homework Title}
\rhead{\today}
\fancyfoot[C]{\scriptsize Page \thepage\ of \pageref*{LastPage} \textbar{} Licensed under the GNU GPL v3.0 or later}

% Custom mathematical commands
\newcommand{\R}{\mathbb{R}}
\newcommand{\F}{\mathbb{F}}
\newcommand{\LL}{\mathcal{L}}
\newcommand{\C}{\mathbb{C}}
\newcommand{\spn}{\operatorname{span}}
\newcommand{\nll}{\operatorname{null}}
\newcommand{\range}{\operatorname{range}}

% Theorems and definitions
\newtheorem{theorem}{Theorem}
\newtheorem{lemma}[theorem]{Lemma}
\newtheorem{proposition}[theorem]{Proposition}
\newtheorem{corollary}[theorem]{Corollary}
\newtheorem{definition}[theorem]{Definition}
\newtheorem{example}[theorem]{Example}
\newtheorem{exercise}[theorem]{Exercise}
\newtheorem{remark}[theorem]{Remark}

\begin{document}

\maketitle

\begin{abstract}
  This is a template for homework assignments. Replace this abstract with a brief summary of your work or remove it entirely if not needed.
\end{abstract}

\section{Problem 1}

State the problem here.

\subsection{Solution}

Provide your solution here.

\begin{theorem}
  State any theorems you prove.
\end{theorem}

\begin{proof}
  Provide your proof here.
\end{proof}

\subsection{Code Implementation}

You can use either minted or listings for code blocks:

\subsubsection{Using minted (requires pygments)}

\begin{minted}[linenos, frame=lines]{python}
def example_function():
    """
    Example function for demonstration.
    """
    return "Hello, World!"
\end{minted}

\subsubsection{Using listings}

\begin{lstlisting}[caption={Example Python Code}, label={lst:example}]
def example_function():
    """
    Example function for demonstration.
    """
    return "Hello, World!"
\end{lstlisting}

\section{Problem 2}

State the next problem here.

\subsection{Mathematical Equations}

You can include inline math like $f(x) = x^2$ or display math:

\[
\int_{-\infty}^{\infty} e^{-x^2} dx = \sqrt{\pi}
\]

\subsection{Figures}

\begin{figure}[H]
  \centering
  % \includegraphics[width=0.5\textwidth]{your-figure.png}
  \caption{Caption for your figure}
  \label{fig:example}
\end{figure}

\subsection{Tables}

\begin{table}[H]
  \centering
  \begin{tabular}{|c|c|c|}
    \hline
    \textbf{Column 1} & \textbf{Column 2} & \textbf{Column 3} \\ \hline
    Data 1 & Data 2 & Data 3 \\ \hline
    Data 4 & Data 5 & Data 6 \\ \hline
  \end{tabular}
  \caption{Example table}
  \label{tab:example}
\end{table}

\section{Problem 3}

\subsection{Algorithms}

\begin{algorithm}[H]
\caption{Example Algorithm}
\begin{algorithmic}[1]
\Procedure{ExampleAlgorithm}{$input$}
    \State $result \gets 0$
    \For{$i \gets 1$ to $n$}
        \State $result \gets result + i$
    \EndFor
    \State \textbf{return} $result$
\EndProcedure
\end{algorithmic}
\end{algorithm}

\section{Conclusion}

Summarize your work here.

\section*{License}

This document is licensed under the \textbf{GNU General Public License v3.0 or later}. You are free to copy, modify, and distribute it under the same terms. For the full license, visit \url{https://www.gnu.org/licenses/gpl-3.0.en.html}.

% Uncomment the following line if you have a bibliography
% \printbibliography

\end{document} 